% mnras_template.tex
%
% LaTeX template for creating an MNRAS paper
%
% v3.0 released 14 May 2015
% (version numbers match those of mnras.cls)
%
% Copyright (C) Royal Astronomical Society 2015
% Authors:
% Keith T. Smith (Royal Astronomical Society)

% Change log
%
% v3.0 May 2015
%    Renamed to match the new package name
%    Version number matches mnras.cls
%    A few minor tweaks to wording
% v1.0 September 2013
%    Beta testing only - never publicly released
%    First version: a simple (ish) template for creating an MNRAS paper

%%%%%%%%%%%%%%%%%%%%%%%%%%%%%%%%%%%%%%%%%%%%%%%%%%
% Basic setup. Most papers should leave these options alone.
\documentclass[fleqn,usenatbib]{mnras}

% MNRAS is set in Times font. If you don't have this installed (most LaTeX
% installations will be fine) or prefer the old Computer Modern fonts, comment
% out the following line
%\usepackage{newtxtext,newtxmath}
\usepackage{amsfonts}
% Depending on your LaTeX fonts installation, you might get better results with one of these:
%\usepackage{mathptmx}
%\usepackage{txfonts}

% Use vector fonts, so it zooms properly in on-screen viewing software
% Don't change these lines unless you know what you are doing
\usepackage[T1]{fontenc}
\usepackage{ae,aecompl}

%%%%% AUTHORS - PLACE YOUR OWN PACKAGES HERE %%%%%

%\usepackage{morefloats}% A pacakge which enables this text to contain more figures and tables.
\usepackage{grffile}% A pacakge which changes the algorithm to check for known extensions, so that we can insert pdf figures into this text.

% Only include extra packages if you really need them. Common packages are:
\usepackage{graphicx}	% Including figure files
\usepackage{amsmath}	% Advanced maths commands
\usepackage{amssymb}	% Extra maths symbols

%%%%%%%%%%%%%%%%%%%%%%%%%%%%%%%%%%%%%%%%%%%%%%%%%%

%%%%% AUTHORS - PLACE YOUR OWN COMMANDS HERE %%%%%

% Please keep new commands to a minimum, and use \newcommand not \def to avoid
% overwriting existing commands. Example:
%\newcommand{\pcm}{\,cm$^{-2}$}	% per cm-squared

%%%%%%%%%%%%%%%%%%%%%%%%%%%%%%%%%%%%%%%%%%%%%%%%%%

%%%%%%%%%%%%%%%%%%% TITLE PAGE %%%%%%%%%%%%%%%%%%%

% Title of the paper, and the short title which is used in the headers.
% Keep the title short and informative.
\title[Homework 2]{Homework 2 for Introduction to Observational Cosmology}

% The list of authors, and the short list which is used in the headers.
% If you need two or more lines of authors, add an extra line using \newauthor
\author[Bo-Yuan Liu]{Bo-Yuan Liu$^{1}$\thanks{E-mail: liu-by13@mails.tsinghua.edu.cn}
\\
% List of institutions
$^{1}$Department of Physics, Tsinghua University, Beijing, China(PRC)\\
}

% These dates will be filled out by the publisher
\date{Accepted XXX. Received YYY; in original form ZZZ}

% Enter the current year, for the copyright statements etc.
\pubyear{2016}

% Don't change these lines
\begin{document}
\label{firstpage}
\pagerange{\pageref{firstpage}--\pageref{lastpage}}
\maketitle

% Abstract of the paper
\begin{abstract}
I selected a universe with $\Omega_{R,0}=1$, $\Omega_{\Lambda,0}=1$, $\Omega_{M,0}=0$, and $\Omega_{C,0}=(1-1-1)=-1$, which means that we are now at the very moment when ``the light'' and ``the darkness'' are in balance for a positively curved space without any matter. From Friedman equations, I calculated analytically the age $t(a)$, the horizon $\chi_{hor}(a)$ and the deceleration parameter $q(a)$. The results for the present are $t_{0}=t(1)\sim 8\mathrm{Gyr}$, and $\chi_{hor,0}=\chi_{hor}(1)\sim 4.8\mathrm{Gpc}$. It is shown that this universe expands continuously, and we are at the very point where it shifts from decelerating expansion to accelerating expansion ($\ddot{a}=0$). I did not check my results on the given website in that it cannot include radiation. I also calculated several luminosity distances under different redshifts, and plotted the $d_{L}-z$, $d_{A}-z$ and $d_{M}-z$ diagrams to illustrate the relations among the luminosity distance $d_{L}$, the angular diameter distance $d_{A}$, the proper motion distance $d_{M}$ and the redshit $z$.
\end{abstract}

% Select between one and six entries from the list of approved keywords.
% Don't make up new ones.
\begin{keywords}
Friedman equations -- universe: age, horizon, deceleration parameter
\end{keywords}

%%%%%%%%%%%%%%%%%%%%%%%%%%%%%%%%%%%%%%%%%%%%%%%%%%

%%%%%%%%%%%%%%%%% BODY OF PAPER %%%%%%%%%%%%%%%%%%
 
\section{General formulas}


\bibliographystyle{mnras}
\bibliography{ref} 

\label{lastpage}
\end{document}