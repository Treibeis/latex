\documentclass[cjk,slidestop,compress,mathserif]{beamer}
\usepackage[utf8]{inputenc}
\usetheme{JuanLesPins}
\usepackage{ctex}
\newcommand{\ud}{\mathrm{d}}%非斜体的d
\newcommand{\uln}{\mathrm{ln}}
\usecolortheme{seahorse}
\usepackage{amsmath}
\usepackage{amsfonts}
\usepackage{amssymb}
\usepackage{bm}
\author{刘博远, 2013012221\\ 物理 32, 清华大学物理系}
\title{磁控溅射镀膜}
\begin{document}
%\beamertemplatetransparentcovereddynamicmedium
\begin{frame}
\titlepage
\end{frame}
\section{基本原理}
\begin{frame}
\frametitle{气体辉光放电}

\begin{itemize}
\begin{columns}

\begin{column}{0.65\textwidth}
\pause \item 击穿电压(Paschen's law$^{[2]}$)\\
\pause $V_{b}=\frac{a\cdot pd}{b+\uln(pd)}$\\
其中$a$、$b$是和气体、电极性质有关的常数,$p$是气体压强 
\pause \item 正常辉光放电($0.1\sim 10$Pa Ar气)
\pause \item 反常辉光放电(溅射区):\\
\pause $U=E+Fj/p$\\
其中$F$和$E$是取决于电极材料、几何尺寸和气体成分的常数,$j$是电流密度
\end{column}

\begin{column}{0.35\textwidth}
\begin{figure}
\centering
\includegraphics[width=0.9\textwidth]{charge.png}
\caption{\begin{tiny}
气体放电的$U-I$曲线$^{[3]}$
\end{tiny}}
\end{figure}
\end{column}
\end{columns}
\end{itemize}
\end{frame}

\section{Thanks for your attention}
\begin{frame}
\centering
\large{Thanks for your attention}
\begin{itemize}
\item 参考文献
\begin{itemize}
\item $[1]$ 实验讲义. \\
\item $[2]$ Paschen's law, wikipedia.
\item $[3]$ Glow discharge, wikipedia. 
\end{itemize}
\end{itemize}
\end{frame}
\end{document}