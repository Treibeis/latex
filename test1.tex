\documentclass[10pt,a4paper]{article}
\usepackage[utf8]{inputenc}
\usepackage{amsmath}
\usepackage{amsfonts}
\usepackage{amssymb}
\author{Liu Boyuan}
\usepackage{CJK}
\usepackage{textcomp}
\usepackage{verbatim}
\usepackage{indentfirst}
\usepackage{graphicx}
\usepackage{etoolbox}
\usepackage{bm}%加粗
\usepackage{subfigure}
\newcommand{\ud}{\mathrm{d}}%斜体的d
\makeatletter
\let\@afterindentfalse\@afterindenttrue
\@afterindenttrue
\makeatother
\setlength{\parindent}{2em}
\begin{document}
\part{An article}
\begin{CJK}{UTF8}{song}
我勒个去。\\
\end{CJK}
\emph{It doesn't work.}\\
\LaTeX\\
Today is \today.
\textsl{ABC}%斜体
$\backslash$%斜杠
\texteuro\textdollar\textyen \\
This is ABCD\@. Here we start another sentence.\\
\label{mark1} %标记
\section*{nothing}%不占标号
see section\ref{mark1} on page\pageref{mark1}.
\section{footnote}%脚注
Footnotes\footnote{This is a footnote.} are often used.\\

\section{environment}

\begin{enumerate}
\item[]item 1
\end{enumerate}

\begin{flushleft}
asdadsadasdsa.
\end{flushleft}

%quote abstract verse 

\begin{tabular}{c@{***}c|c} %@{}设置分隔字符
\hline %表顶/分隔线
Red&Blue&yellow\\
\hline
White&Black&Gray\\
\hline
\multicolumn{2}{c}{1\&2}&3\\ %合并单元格
\hline
\end{tabular}
\center{asasasa}

\begin{displaymath}%无标号数学环境
a^{2}+b^{2}=c^{2}\textrm{haha}%添加文字
\end{displaymath}
\[
1+1=2 \quad \binom{n}{m} %矩阵括号
 \quad  \sqrt[3]{27}=3 \qquad \overline{x}
\]
\begin{flushleft}

Why is it in the center? $\int_{E}f(x)$\\
$x \qquad \bm y \quad z$ %空格\bm粗体
\[
\left(\begin{array}{ccc}
x_{11}&x_{12}&x_{13}\\
m&n&\ldots\\
\vdots&\vdots&\ddots
\end{array}\right)
\]
\begin{equation}
\lim_{x\to\infty}\sum_{n=1}^{\infty}\frac{1}{n^{2}}=\frac{\pi^{2}}{6}
\end{equation}

\begin{equation}
\ud F=0, \quad \ud\left(*F\right)=*J, \quad F=\ud A, \quad A^{\mu}=\left(\phi,\overrightarrow{A}\right), \quad J^{\mu}=\left(\rho, \overrightarrow{j}\right)
\end{equation}

\begin{eqnarray} %按=对齐
g(x)&=&\sin{x} \notag \\
h(x)&=&\cos{x}
\end{eqnarray}

$\Delta \epsilon_{i} \ll k_{B}T$

\begin{figure}[!htp] %不加htp图片位置不确定
\centering
\includegraphics[width=0.5\textwidth]{figure1}
\caption{Fig.1}
\end{figure}

% Table generated by Excel2LaTeX from sheet 'Sheet1'
\begin{table}[htbp]
  \centering
  \caption{Add caption}
    \begin{tabular}{rrr}
    \hline
    N     & U     & I \\
    \hline
    1     & 0.048 & 10 \\
    2     & 0.06  & 20 \\
    3     & 0.068 & 30 \\
    4     & 0.07  & 40 \\
    5     & 0.091 & 50 \\
    6     & 0.13  & 60 \\
    7     & 0.131 & 70 \\
    8     & 0.109 & 80 \\
    9     & 0.116 & 82 \\
    10    & 0.124 & 84 \\
    11    & 0.126 & 86 \\
    12    & 0.112 & 88 \\
    13    & 0.066 & 90 \\
    14    & 0.033 & 92 \\
    15    & 0.016 & 94 \\
    16    & 0.098 & 96 \\
    17    & 0.133 & 97 \\
    18    & 0.189 & 98 \\
    19    & 0.267 & 99 \\
    20    & 0.345 & 100 \\
    21    & 0.415 & 101 \\
    22    & 0.51  & 102 \\
    23    & 0.676 & 103 \\
    24    & 0.847 & 104 \\
    25    & 1.088 & 105 \\
    26    & 1.388 & 106 \\
    27    & 1.56  & 107 \\
    28    & 1.962 & 108 \\
    29    & 2.15  & 109 \\
    30    & 2.6   & 110 \\
    31    & 3.043 & 111 \\
    \hline
    \end{tabular}%
  \label{tab:addlabel}%
\end{table}%


\end{flushleft}
\end{document}