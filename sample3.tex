\documentclass[10.5pt,a4paper]{ctexart}
\usepackage[utf8]{inputenc}
\usepackage{geometry}
\usepackage{amsmath}
\usepackage{amsfonts}
\usepackage{amssymb}
\usepackage{xeCJK}
\usepackage{textcomp}
\usepackage{verbatim}
\usepackage{indentfirst}
\usepackage{graphicx}
\usepackage{etoolbox}
\usepackage{bm}%加粗
\usepackage{subfigure}
\newcommand{\ud}{\mathrm{d}}%非斜体的d
\newcommand{\uln}{\mathrm{ln}}
\geometry{right=3.18cm,left=3.18cm,top=2.54cm,bottom=2.54cm} %word常用的页边距
\begin{document}
\fontsize{10.5pt}{15.75pt}\selectfont %五号、1.5倍行距
\title{磁控溅射镀膜\\
\begin{normalsize}
刘博远 2013012221 物理32\\指导教师:王合英 \quad 实验日期:2016年4月14日
\end{normalsize}
}
\date{}
\author{}
\maketitle

\section*{【摘要】}
本实验以金属铜为靶材、氩气为放电气体、载玻片为基片,研究平衡直流磁控溅射中薄膜生长速率和气体压强以及放电电流、功率的关系。因为数据点太少和一些读数错误,本实验的结果虽然大致和理论解释相符,但是并不能得出有意义的结论。于是我们总结了经验教训,并对对今后的实验提出了建议。\\
\indent 关键词:磁控溅射、薄膜生长速率、气体放电:压强、电流、电压、功率
\section{参考文献}
\noindent
$[1]$ 清华大学近代物理实验室, 磁控溅射镀膜实验讲义, unpublished.\\
$[2]$ 何万强等, 非平衡磁控溅射系统的研究, 新技术新工艺 机械加工与自动化, No.6, 2002.\\
$[3]$ 赵嘉学, 童洪辉, 磁控溅射原理的深入探讨, 真空, Vol.41, No.4, 2004.\\
$[4]$ 王璘等, 磁控溅射镀膜中工作气压对沉积速率的影响, 真空, Vol.41, No.1, 2004.\\
$[5]$ Glow discharge, wikipedia. 
\end{document}
